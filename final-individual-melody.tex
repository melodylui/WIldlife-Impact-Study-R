% Options for packages loaded elsewhere
\PassOptionsToPackage{unicode}{hyperref}
\PassOptionsToPackage{hyphens}{url}
%
\documentclass[
]{article}
\usepackage{amsmath,amssymb}
\usepackage{iftex}
\ifPDFTeX
  \usepackage[T1]{fontenc}
  \usepackage[utf8]{inputenc}
  \usepackage{textcomp} % provide euro and other symbols
\else % if luatex or xetex
  \usepackage{unicode-math} % this also loads fontspec
  \defaultfontfeatures{Scale=MatchLowercase}
  \defaultfontfeatures[\rmfamily]{Ligatures=TeX,Scale=1}
\fi
\usepackage{lmodern}
\ifPDFTeX\else
  % xetex/luatex font selection
\fi
% Use upquote if available, for straight quotes in verbatim environments
\IfFileExists{upquote.sty}{\usepackage{upquote}}{}
\IfFileExists{microtype.sty}{% use microtype if available
  \usepackage[]{microtype}
  \UseMicrotypeSet[protrusion]{basicmath} % disable protrusion for tt fonts
}{}
\makeatletter
\@ifundefined{KOMAClassName}{% if non-KOMA class
  \IfFileExists{parskip.sty}{%
    \usepackage{parskip}
  }{% else
    \setlength{\parindent}{0pt}
    \setlength{\parskip}{6pt plus 2pt minus 1pt}}
}{% if KOMA class
  \KOMAoptions{parskip=half}}
\makeatother
\usepackage{xcolor}
\usepackage[margin=1in]{geometry}
\usepackage{color}
\usepackage{fancyvrb}
\newcommand{\VerbBar}{|}
\newcommand{\VERB}{\Verb[commandchars=\\\{\}]}
\DefineVerbatimEnvironment{Highlighting}{Verbatim}{commandchars=\\\{\}}
% Add ',fontsize=\small' for more characters per line
\usepackage{framed}
\definecolor{shadecolor}{RGB}{248,248,248}
\newenvironment{Shaded}{\begin{snugshade}}{\end{snugshade}}
\newcommand{\AlertTok}[1]{\textcolor[rgb]{0.94,0.16,0.16}{#1}}
\newcommand{\AnnotationTok}[1]{\textcolor[rgb]{0.56,0.35,0.01}{\textbf{\textit{#1}}}}
\newcommand{\AttributeTok}[1]{\textcolor[rgb]{0.13,0.29,0.53}{#1}}
\newcommand{\BaseNTok}[1]{\textcolor[rgb]{0.00,0.00,0.81}{#1}}
\newcommand{\BuiltInTok}[1]{#1}
\newcommand{\CharTok}[1]{\textcolor[rgb]{0.31,0.60,0.02}{#1}}
\newcommand{\CommentTok}[1]{\textcolor[rgb]{0.56,0.35,0.01}{\textit{#1}}}
\newcommand{\CommentVarTok}[1]{\textcolor[rgb]{0.56,0.35,0.01}{\textbf{\textit{#1}}}}
\newcommand{\ConstantTok}[1]{\textcolor[rgb]{0.56,0.35,0.01}{#1}}
\newcommand{\ControlFlowTok}[1]{\textcolor[rgb]{0.13,0.29,0.53}{\textbf{#1}}}
\newcommand{\DataTypeTok}[1]{\textcolor[rgb]{0.13,0.29,0.53}{#1}}
\newcommand{\DecValTok}[1]{\textcolor[rgb]{0.00,0.00,0.81}{#1}}
\newcommand{\DocumentationTok}[1]{\textcolor[rgb]{0.56,0.35,0.01}{\textbf{\textit{#1}}}}
\newcommand{\ErrorTok}[1]{\textcolor[rgb]{0.64,0.00,0.00}{\textbf{#1}}}
\newcommand{\ExtensionTok}[1]{#1}
\newcommand{\FloatTok}[1]{\textcolor[rgb]{0.00,0.00,0.81}{#1}}
\newcommand{\FunctionTok}[1]{\textcolor[rgb]{0.13,0.29,0.53}{\textbf{#1}}}
\newcommand{\ImportTok}[1]{#1}
\newcommand{\InformationTok}[1]{\textcolor[rgb]{0.56,0.35,0.01}{\textbf{\textit{#1}}}}
\newcommand{\KeywordTok}[1]{\textcolor[rgb]{0.13,0.29,0.53}{\textbf{#1}}}
\newcommand{\NormalTok}[1]{#1}
\newcommand{\OperatorTok}[1]{\textcolor[rgb]{0.81,0.36,0.00}{\textbf{#1}}}
\newcommand{\OtherTok}[1]{\textcolor[rgb]{0.56,0.35,0.01}{#1}}
\newcommand{\PreprocessorTok}[1]{\textcolor[rgb]{0.56,0.35,0.01}{\textit{#1}}}
\newcommand{\RegionMarkerTok}[1]{#1}
\newcommand{\SpecialCharTok}[1]{\textcolor[rgb]{0.81,0.36,0.00}{\textbf{#1}}}
\newcommand{\SpecialStringTok}[1]{\textcolor[rgb]{0.31,0.60,0.02}{#1}}
\newcommand{\StringTok}[1]{\textcolor[rgb]{0.31,0.60,0.02}{#1}}
\newcommand{\VariableTok}[1]{\textcolor[rgb]{0.00,0.00,0.00}{#1}}
\newcommand{\VerbatimStringTok}[1]{\textcolor[rgb]{0.31,0.60,0.02}{#1}}
\newcommand{\WarningTok}[1]{\textcolor[rgb]{0.56,0.35,0.01}{\textbf{\textit{#1}}}}
\usepackage{graphicx}
\makeatletter
\def\maxwidth{\ifdim\Gin@nat@width>\linewidth\linewidth\else\Gin@nat@width\fi}
\def\maxheight{\ifdim\Gin@nat@height>\textheight\textheight\else\Gin@nat@height\fi}
\makeatother
% Scale images if necessary, so that they will not overflow the page
% margins by default, and it is still possible to overwrite the defaults
% using explicit options in \includegraphics[width, height, ...]{}
\setkeys{Gin}{width=\maxwidth,height=\maxheight,keepaspectratio}
% Set default figure placement to htbp
\makeatletter
\def\fps@figure{htbp}
\makeatother
\setlength{\emergencystretch}{3em} % prevent overfull lines
\providecommand{\tightlist}{%
  \setlength{\itemsep}{0pt}\setlength{\parskip}{0pt}}
\setcounter{secnumdepth}{-\maxdimen} % remove section numbering
\ifLuaTeX
  \usepackage{selnolig}  % disable illegal ligatures
\fi
\IfFileExists{bookmark.sty}{\usepackage{bookmark}}{\usepackage{hyperref}}
\IfFileExists{xurl.sty}{\usepackage{xurl}}{} % add URL line breaks if available
\urlstyle{same}
\hypersetup{
  pdftitle={The Assumptions of Bigfoot},
  pdfauthor={Melody Lui},
  hidelinks,
  pdfcreator={LaTeX via pandoc}}

\title{The Assumptions of Bigfoot}
\author{Melody Lui}
\date{}

\begin{document}
\maketitle

\hypertarget{objectives}{%
\subsection{Objectives}\label{objectives}}

To collect all the thoughts and knowledge from the past semester and tie
it into a single assignment. As well as grow out knowledge on the tools
of R.

\hypertarget{introduction}{%
\subsection{Introduction}\label{introduction}}

Journalist, Laura Krantz, wrote an article for \emph{High Country News}
in 2019 titled {[}`Even if Bigfoot isn't real, we still need him'{]}
(\url{https://www.hcn.org/articles/essays-even-if-bigfoot-isnt-real-we-still-need-him})
after spending two years researching and tracking down the ins and outs
of Bigfoot.

An interesting claim that Laura makes, considering that Bigfoot is `just
a myth' in American society.

As the article progresses, Laura eludes to the idea of how American
culture is dissociated from nature and the Earth. Humans, especially
Americans, have had a decreased sense of engagement with the environment
as technology continues to increase and society continues to move
towards a complete isolation from all things natural.

Laura continues to explain her cause, even talking to several biologists
and researchers about Bigfoot. The main reason for trying to discover
such a myth as Bigfoot is to preserve the area its concentrated around.
John Kirk, President of the British Columbia Scientific Cryptozoology
Club, said to Laura, ``I think habitat's worth preserving plain and
simple, but if you can put a biological rarity into that equation like
they did with the spotted owl\ldots{}'' then preservation becomes even
more important.

Therefore the hunt for Bigfoot is vaguely about catching a rare and
mythical creature that has yet to be proven real, but moreso about
preserving the little nature that remains on our Earth. It is moreso
about rekindling our relationship with nature, organisms, and the Earth
and being able to connect with what we keep diminishing.

\begin{Shaded}
\begin{Highlighting}[]
\NormalTok{bigfoot\_sightings }\OtherTok{\textless{}{-}} \FunctionTok{read\_csv}\NormalTok{(}\StringTok{"data/bfro\_reports\_geocoded.csv"}\NormalTok{, }\AttributeTok{col\_types =} \StringTok{"ccccccddDdccdddddddddcdccdddd"}\NormalTok{)}
\NormalTok{bigfoot\_sightings}
\end{Highlighting}
\end{Shaded}

\begin{verbatim}
## # A tibble: 5,082 x 29
##    observed        location_details county state season title latitude longitude
##    <chr>           <chr>            <chr>  <chr> <chr>  <chr>    <dbl>     <dbl>
##  1 "I am not sure~ "We were on our~ Washa~ Wyom~ Summer  <NA>     NA        NA  
##  2 "I don't know ~ "Heading to the~ Wyomi~ West~ Winter "Rep~     37.6     -81.3
##  3 "My family and~ "It's off Rt 10~ Winds~ Verm~ Fall   "Rep~     43.5     -72.7
##  4 "It was spring~ "Wythe county V~ Wythe~ Virg~ Spring "Rep~     37.2     -81.1
##  5 "It was the wi~ "Hwy 182, Wood ~ Wood ~ Texas Winter "Rep~     32.8     -95.5
##  6 "My brother an~ "Sighting occur~ Willi~ Tenn~ Spring  <NA>     NA        NA  
##  7 "This happened~ "York South Car~ York ~ Sout~ Fall    <NA>     NA        NA  
##  8 "While attendi~ "Great swamp ar~ Washi~ Rhod~ Fall   "Rep~     41.4     -71.5
##  9 "Hello, My nam~ "I would rather~ York ~ Penn~ Summer  <NA>     NA        NA  
## 10 "It was May 19~ "Logging roads ~ Yamhi~ Oreg~ Spring  <NA>     NA        NA  
## # i 5,072 more rows
## # i 21 more variables: date <date>, number <dbl>, classification <chr>,
## #   geohash <chr>, temperature_high <dbl>, temperature_mid <dbl>,
## #   temperature_low <dbl>, dew_point <dbl>, humidity <dbl>, cloud_cover <dbl>,
## #   moon_phase <dbl>, precip_intensity <dbl>, precip_probability <dbl>,
## #   precip_type <chr>, pressure <dbl>, summary <chr>, conditions <chr>,
## #   uv_index <dbl>, visibility <dbl>, wind_bearing <dbl>, wind_speed <dbl>
\end{verbatim}

\hypertarget{part-1}{%
\subsection{Part 1}\label{part-1}}

Bigfeet are the most seen in the Pacific Northwest near the western
coast of the United States. We want to focus all of our data into the
Pacific Northwest to make things easier to run. After creating smaller
data frames, we will want to determine the best, most ideal conditions
that Bigfeet live in. Assume that the Bigfoot Sightings are the exact
location of Bigfoot habitat.

\begin{Shaded}
\begin{Highlighting}[]
\NormalTok{bigfoot\_wo }\OtherTok{\textless{}{-}} \FunctionTok{filter}\NormalTok{(bigfoot\_sightings, state }\SpecialCharTok{==} \StringTok{"Washington"} \SpecialCharTok{|}\NormalTok{ state }\SpecialCharTok{==} \StringTok{"Oregon"}\NormalTok{) }

\NormalTok{bigfoot\_cal }\OtherTok{\textless{}{-}} \FunctionTok{filter}\NormalTok{(bigfoot\_sightings, state }\SpecialCharTok{==} \StringTok{"California"}\NormalTok{)}
                      
\NormalTok{bigfoot\_pnwcal }\OtherTok{\textless{}{-}} \FunctionTok{filter}\NormalTok{(bigfoot\_cal, county }\SpecialCharTok{==} \StringTok{"Siskiyou County"}\SpecialCharTok{|}\NormalTok{ county }\SpecialCharTok{==} \StringTok{"Del Norte County"}\SpecialCharTok{|}\NormalTok{ county }\SpecialCharTok{==} \StringTok{"Humboldt County"}\NormalTok{) }

\NormalTok{bigfoot\_pnw }\OtherTok{\textless{}{-}} \FunctionTok{bind\_rows}\NormalTok{(bigfoot\_wo, bigfoot\_pnwcal)}
\end{Highlighting}
\end{Shaded}

\begin{Shaded}
\begin{Highlighting}[]
\NormalTok{pnw\_sum }\OtherTok{\textless{}{-}} \FunctionTok{filter}\NormalTok{(bigfoot\_pnw, season }\SpecialCharTok{==} \StringTok{"Summer"}\NormalTok{)}
\NormalTok{pnw\_fall }\OtherTok{\textless{}{-}} \FunctionTok{filter}\NormalTok{(bigfoot\_pnw, season }\SpecialCharTok{==} \StringTok{"Fall"}\NormalTok{)}
\NormalTok{pnw\_winter }\OtherTok{\textless{}{-}} \FunctionTok{filter}\NormalTok{(bigfoot\_pnw, season }\SpecialCharTok{==} \StringTok{"Winter"}\NormalTok{)}
\NormalTok{pnw\_spring }\OtherTok{\textless{}{-}} \FunctionTok{filter}\NormalTok{(bigfoot\_pnw, season }\SpecialCharTok{==} \StringTok{"Spring"}\NormalTok{)}

\NormalTok{pnw\_count }\OtherTok{\textless{}{-}}\NormalTok{ bigfoot\_pnw }\SpecialCharTok{\%\textgreater{}\%} \FunctionTok{count}\NormalTok{(season)}

\NormalTok{szn\_graph }\OtherTok{\textless{}{-}} \FunctionTok{ggplot}\NormalTok{(bigfoot\_pnw) }\SpecialCharTok{+} \FunctionTok{geom\_bar}\NormalTok{(}\FunctionTok{aes}\NormalTok{(}\AttributeTok{x =}\NormalTok{ season))}
\NormalTok{szn\_graph}
\end{Highlighting}
\end{Shaded}

\includegraphics{final-individual-melody_files/figure-latex/seasons-1.pdf}

Bigfoot are usually seen in the Summer seasons in the Pacific Northwest.
This may be because it is usually more clear and more daylight impedes
on the forests of the PNW in the summertime, however plots of the
weather conditions in the summer show that it is actually almost equally
likely that it will be clear or partially cloudy.

This graph is important as it can also be explained by the amount of CO2
in the atmosphere during these times. CO2 is in abundance more often in
the Springtime as the CO2 has accumulated during the winter seasons.
During the summertime, the CO2 is slowly escaping and is at its all time
low around October after the summer. As Bigfeet are mammals, they could
lack oxygen even in highly forested areas. Bigfeet are seen to be large
mammalian creatures, they would need to intake a lot of oxygen just to
survive. The high amounts of CO2 during the Winter could lead to
hibernation like bears. As the Spring and Summer roll around, they are
able to breathe and find energy to fight through the several months.

\begin{Shaded}
\begin{Highlighting}[]
\NormalTok{moon\_phase\_sum }\OtherTok{\textless{}{-}} \FunctionTok{ggplot}\NormalTok{(pnw\_sum) }\SpecialCharTok{+} \FunctionTok{geom\_bar}\NormalTok{(}\FunctionTok{aes}\NormalTok{(}\AttributeTok{x =}\NormalTok{ moon\_phase))}
\NormalTok{moon\_phase\_sum}
\end{Highlighting}
\end{Shaded}

\begin{verbatim}
## Warning: Removed 87 rows containing non-finite values (`stat_count()`).
\end{verbatim}

\includegraphics{final-individual-melody_files/figure-latex/moon graph-1.pdf}

\begin{Shaded}
\begin{Highlighting}[]
\NormalTok{moon\_phase }\OtherTok{\textless{}{-}} \FunctionTok{ggplot}\NormalTok{(bigfoot\_pnw) }\SpecialCharTok{+} \FunctionTok{geom\_bar}\NormalTok{(}\FunctionTok{aes}\NormalTok{(}\AttributeTok{x =}\NormalTok{ moon\_phase))}
\NormalTok{moon\_phase}
\end{Highlighting}
\end{Shaded}

\begin{verbatim}
## Warning: Removed 183 rows containing non-finite values (`stat_count()`).
\end{verbatim}

\includegraphics{final-individual-melody_files/figure-latex/moon graph-2.pdf}

Moon phase could be important in figuring out when Bigfoot come out the
most if they are more supernatural like Werewolves. Werewolves are most
often known to come out during full moons. However, we see that the
Bigfoot are usually spotted during the third quarter of the moon phase,
also known as the Waning Gibbous. I am performing both summer only
Bigfoot sightings as well as all the PNW sightings as this will help see
if there is a pattern between ALL Bigfoot sightings or just the summer
months.

\begin{Shaded}
\begin{Highlighting}[]
\NormalTok{weather\_sum }\OtherTok{\textless{}{-}} \FunctionTok{ggplot}\NormalTok{(pnw\_sum) }\SpecialCharTok{+} \FunctionTok{geom\_bar}\NormalTok{(}\FunctionTok{aes}\NormalTok{(}\AttributeTok{x =}\NormalTok{ conditions)) }\SpecialCharTok{+}
    \FunctionTok{theme}\NormalTok{(}\AttributeTok{text =} \FunctionTok{element\_text}\NormalTok{(}\AttributeTok{size=}\DecValTok{10}\NormalTok{),}
        \AttributeTok{axis.text.x =} \FunctionTok{element\_text}\NormalTok{(}\AttributeTok{angle=}\DecValTok{45}\NormalTok{, }\AttributeTok{hjust=}\DecValTok{1}\NormalTok{)) }
\NormalTok{weather\_sum}
\end{Highlighting}
\end{Shaded}

\includegraphics{final-individual-melody_files/figure-latex/weather conditions-1.pdf}

\begin{Shaded}
\begin{Highlighting}[]
\NormalTok{weather }\OtherTok{\textless{}{-}} \FunctionTok{ggplot}\NormalTok{(bigfoot\_pnw) }\SpecialCharTok{+} \FunctionTok{geom\_bar}\NormalTok{(}\FunctionTok{aes}\NormalTok{(}\AttributeTok{x =}\NormalTok{ conditions)) }\SpecialCharTok{+}
    \FunctionTok{theme}\NormalTok{(}\AttributeTok{text =} \FunctionTok{element\_text}\NormalTok{(}\AttributeTok{size=}\DecValTok{10}\NormalTok{),}
        \AttributeTok{axis.text.x =} \FunctionTok{element\_text}\NormalTok{(}\AttributeTok{angle=}\DecValTok{45}\NormalTok{, }\AttributeTok{hjust=}\DecValTok{1}\NormalTok{)) }
\NormalTok{weather}
\end{Highlighting}
\end{Shaded}

\includegraphics{final-individual-melody_files/figure-latex/weather conditions-2.pdf}

\begin{Shaded}
\begin{Highlighting}[]
\NormalTok{vis\_sum }\OtherTok{\textless{}{-}} \FunctionTok{ggplot}\NormalTok{(pnw\_sum) }\SpecialCharTok{+} \FunctionTok{geom\_bar}\NormalTok{(}\FunctionTok{aes}\NormalTok{(}\AttributeTok{x =}\NormalTok{ visibility))}
\NormalTok{vis\_sum}
\end{Highlighting}
\end{Shaded}

\begin{verbatim}
## Warning: Removed 102 rows containing non-finite values (`stat_count()`).
\end{verbatim}

\includegraphics{final-individual-melody_files/figure-latex/visibility-1.pdf}

\begin{Shaded}
\begin{Highlighting}[]
\NormalTok{visibility }\OtherTok{\textless{}{-}} \FunctionTok{ggplot}\NormalTok{(bigfoot\_pnw) }\SpecialCharTok{+} \FunctionTok{geom\_bar}\NormalTok{(}\FunctionTok{aes}\NormalTok{(}\AttributeTok{x =}\NormalTok{ visibility))}
\NormalTok{visibility}
\end{Highlighting}
\end{Shaded}

\begin{verbatim}
## Warning: Removed 214 rows containing non-finite values (`stat_count()`).
\end{verbatim}

\includegraphics{final-individual-melody_files/figure-latex/visibility-2.pdf}

Visibility is the amount of light and how clear the day was during each
Bigfoot sighting. The higher the visibility, the more clear of a day it
was. Fog, precipitation, and cloudiness can all lower the visibility.
Although most summer days were clear, there were not too many high
visibility days, even throughout the year. Most of the visibility was
around the 9 range, indicating cloudiness or precipitation. This is
evident in the abundance of sightings during cloudy conditions in the
graphs above.

\begin{Shaded}
\begin{Highlighting}[]
\NormalTok{county\_graph }\OtherTok{\textless{}{-}} \FunctionTok{ggplot}\NormalTok{(pnw\_sum, }\FunctionTok{aes}\NormalTok{(}\AttributeTok{x=}\NormalTok{ moon\_phase)) }\SpecialCharTok{+} \FunctionTok{geom\_bar}\NormalTok{() }\SpecialCharTok{+} \FunctionTok{facet\_wrap}\NormalTok{(}\SpecialCharTok{\textasciitilde{}}\NormalTok{ county) }\SpecialCharTok{+}
    \FunctionTok{theme}\NormalTok{(}\AttributeTok{text =} \FunctionTok{element\_text}\NormalTok{(}\AttributeTok{size=}\DecValTok{7}\NormalTok{),}
        \AttributeTok{axis.text.x =} \FunctionTok{element\_text}\NormalTok{(}\AttributeTok{angle=}\DecValTok{90}\NormalTok{, }\AttributeTok{hjust=}\DecValTok{1}\NormalTok{)) }
\NormalTok{county\_graph}
\end{Highlighting}
\end{Shaded}

\begin{verbatim}
## Warning: Removed 87 rows containing non-finite values (`stat_count()`).
\end{verbatim}

\includegraphics{final-individual-melody_files/figure-latex/unnamed-chunk-2-1.pdf}

\begin{Shaded}
\begin{Highlighting}[]
\NormalTok{group }\OtherTok{\textless{}{-}}\NormalTok{ pnw\_sum }\SpecialCharTok{\%\textgreater{}\%} \FunctionTok{count}\NormalTok{(county)}
\NormalTok{group}
\end{Highlighting}
\end{Shaded}

\begin{verbatim}
## # A tibble: 55 x 2
##    county               n
##    <chr>            <int>
##  1 Baker County         1
##  2 Benton County        3
##  3 Chelan County       15
##  4 Clackamas County    17
##  5 Clallam County       5
##  6 Clark County         4
##  7 Clatsop County       4
##  8 Columbia County      3
##  9 Coos County          2
## 10 Cowlitz County      10
## # i 45 more rows
\end{verbatim}

\hypertarget{part-2}{%
\subsection{Part 2}\label{part-2}}

The top 30\% of Bigfoot sightings are within only 6 counties in the
Pacific Northwest: Skamania (Washington), Humboldt (California), Pierce
(Washington), Snohomish (Washington), Siskiyou (California), and
Clackamas (Oregon). The top 25\% is within the first 5 counties out of
55 total counties in the Pacific Northwest. Since these sightings are
not evenly distributed among all 55 counties, there should ideally be a
common theme between the top 6 counties with the most Bigfoot sightings.

As we progress further into our Bigfoot research, California officials
are also looking for Bigfoot as well. The northern-most part of
California has some of the richest biodiversity in the country.
Especially in Siskiyou, the terrain and geography are so unique that the
flora and organisms that reside here are among the most biodiverse.
Siskiyou county, California houses organisms that cannot be found
anywhere else. These organisms must be protected in order to preserve
the Siskiyou county landscape and food webs. Finding a Bigfoot in the
Siskiyou area will garner even more protection and preservation.

The following will focus on California Bigfoot sightings in Humboldt
County (22 Sightings) and Siskiyou County (18 Sightings).

\begin{Shaded}
\begin{Highlighting}[]
\NormalTok{temp }\OtherTok{\textless{}{-}} \FunctionTok{tempfile}\NormalTok{()}
\FunctionTok{download.file}\NormalTok{(}\StringTok{"https://humboldtgov.org/DocumentCenter/View/570"}\NormalTok{, temp)}
\FunctionTok{unzip}\NormalTok{(temp)}
\NormalTok{hum\_sf }\OtherTok{\textless{}{-}} \FunctionTok{read\_sf}\NormalTok{(}\StringTok{"CNTYOUTL.SHP"}\NormalTok{)}

\NormalTok{temp2 }\OtherTok{\textless{}{-}} \FunctionTok{tempfile}\NormalTok{()}
\FunctionTok{download.file}\NormalTok{(}\StringTok{"https://www2.census.gov/geo/tiger/TIGER2020/FACES/tl\_2020\_06093\_faces.zip"}\NormalTok{, temp2)}
\FunctionTok{unzip}\NormalTok{(temp2)}
\NormalTok{sis\_sf }\OtherTok{\textless{}{-}} \FunctionTok{read\_sf}\NormalTok{(}\StringTok{"tl\_2020\_06093\_faces.shp"}\NormalTok{)}
\end{Highlighting}
\end{Shaded}

\begin{Shaded}
\begin{Highlighting}[]
\NormalTok{humboldt\_sightings }\OtherTok{\textless{}{-}} \FunctionTok{filter}\NormalTok{(pnw\_sum, county }\SpecialCharTok{==} \StringTok{"Humboldt County"}\NormalTok{)}

\NormalTok{hum\_latlong }\OtherTok{\textless{}{-}}\NormalTok{ humboldt\_sightings }\SpecialCharTok{\%\textgreater{}\%} \FunctionTok{select}\NormalTok{(longitude, latitude) }\SpecialCharTok{\%\textgreater{}\%} \FunctionTok{na.omit}\NormalTok{()}

\NormalTok{my\_sf }\OtherTok{\textless{}{-}} \FunctionTok{st\_as\_sf}\NormalTok{(hum\_latlong, }\AttributeTok{coords =} \FunctionTok{c}\NormalTok{(}\StringTok{\textquotesingle{}longitude\textquotesingle{}}\NormalTok{, }\StringTok{\textquotesingle{}latitude\textquotesingle{}}\NormalTok{))}

\NormalTok{hum\_plot2 }\OtherTok{\textless{}{-}} \FunctionTok{ggplot}\NormalTok{(my\_sf) }\SpecialCharTok{+} 
  \FunctionTok{geom\_sf}\NormalTok{(}\FunctionTok{aes}\NormalTok{()) }\SpecialCharTok{+} \FunctionTok{labs}\NormalTok{(}\AttributeTok{title =} \StringTok{"Humboldt Sightings"}\NormalTok{, }\AttributeTok{x =} \StringTok{"Longitude"}\NormalTok{, }\AttributeTok{y =} \StringTok{"Latitude"}\NormalTok{) }\SpecialCharTok{+} \FunctionTok{theme}\NormalTok{(}\AttributeTok{text =} \FunctionTok{element\_text}\NormalTok{(}\AttributeTok{size=}\DecValTok{10}\NormalTok{),}
        \AttributeTok{axis.text.x =} \FunctionTok{element\_text}\NormalTok{(}\AttributeTok{angle=}\DecValTok{45}\NormalTok{, }\AttributeTok{hjust=}\DecValTok{1}\NormalTok{)) }

\NormalTok{hum\_plot2}
\end{Highlighting}
\end{Shaded}

\includegraphics{final-individual-melody_files/figure-latex/humboldt sightings-1.pdf}

\begin{Shaded}
\begin{Highlighting}[]
\NormalTok{plot\_hum }\OtherTok{\textless{}{-}} \FunctionTok{ggplot}\NormalTok{() }\SpecialCharTok{+} 
  \FunctionTok{geom\_sf}\NormalTok{(}\AttributeTok{data =}\NormalTok{ hum\_sf, }\AttributeTok{size =} \FloatTok{1.5}\NormalTok{, }\AttributeTok{color =} \StringTok{"black"}\NormalTok{, }\AttributeTok{fill =} \StringTok{"green4"}\NormalTok{) }\SpecialCharTok{+} 
  \FunctionTok{ggtitle}\NormalTok{(}\StringTok{"Humboldt Outline"}\NormalTok{) }\SpecialCharTok{+} 
  \FunctionTok{coord\_sf}\NormalTok{() }\SpecialCharTok{+} \FunctionTok{theme}\NormalTok{(}\AttributeTok{text =} \FunctionTok{element\_text}\NormalTok{(}\AttributeTok{size=}\DecValTok{10}\NormalTok{),}
        \AttributeTok{axis.text.x =} \FunctionTok{element\_text}\NormalTok{(}\AttributeTok{angle=}\DecValTok{45}\NormalTok{, }\AttributeTok{hjust=}\DecValTok{1}\NormalTok{)) }\SpecialCharTok{+} \FunctionTok{labs}\NormalTok{(}\AttributeTok{x =} \StringTok{"Longitude"}\NormalTok{, }\AttributeTok{y =} \StringTok{"Latitude"}\NormalTok{)}
\NormalTok{plot\_hum}
\end{Highlighting}
\end{Shaded}

\includegraphics{final-individual-melody_files/figure-latex/humboldt outline-1.pdf}

\begin{Shaded}
\begin{Highlighting}[]
\NormalTok{hum\_start }\OtherTok{\textless{}{-}} \StringTok{"2022{-}06{-}01"}
  
\NormalTok{hum\_end }\OtherTok{\textless{}{-}} \StringTok{"2022{-}08{-}31"}

\NormalTok{hum\_box }\OtherTok{\textless{}{-}} \FunctionTok{st\_bbox}\NormalTok{(}\FunctionTok{c}\NormalTok{(}\AttributeTok{xmin =} \FloatTok{123.4}\NormalTok{, }\AttributeTok{xmax =} \FloatTok{124.6}\NormalTok{, }\AttributeTok{ymax =} \FloatTok{41.6}\NormalTok{, }\AttributeTok{ymin =} \FloatTok{39.8}\NormalTok{))}

\NormalTok{hum\_items }\OtherTok{\textless{}{-}} 
  \FunctionTok{stac}\NormalTok{(}\StringTok{"https://earth{-}search.aws.element84.com/v0/"}\NormalTok{) }\SpecialCharTok{|\textgreater{}}
  \FunctionTok{stac\_search}\NormalTok{(}\AttributeTok{collections =} \StringTok{"sentinel{-}s2{-}l2a{-}cogs"}\NormalTok{, }\AttributeTok{bbox =} \FunctionTok{c}\NormalTok{(hum\_box), }
              \AttributeTok{datetime =} \FunctionTok{paste}\NormalTok{(hum\_start, hum\_end, }\AttributeTok{sep=}\StringTok{"/"}\NormalTok{),}
              \AttributeTok{limit =} \DecValTok{100}\NormalTok{) }\SpecialCharTok{|\textgreater{}} \FunctionTok{post\_request}\NormalTok{()}

\NormalTok{hum\_col }\OtherTok{\textless{}{-}} \FunctionTok{stac\_image\_collection}\NormalTok{(hum\_items}\SpecialCharTok{$}\NormalTok{features, }\AttributeTok{asset\_names =} \FunctionTok{c}\NormalTok{(}\StringTok{"B02"}\NormalTok{, }\StringTok{"B03"}\NormalTok{, }\StringTok{"B04"}\NormalTok{,}\StringTok{"B08"}\NormalTok{, }\StringTok{"SCL"}\NormalTok{), }\AttributeTok{property\_filter =}\NormalTok{ \textbackslash{}(x) \{x[[}\StringTok{"eo:cloud\_cover"}\NormalTok{]] }\SpecialCharTok{\textless{}} \DecValTok{20}\NormalTok{\})}
\end{Highlighting}
\end{Shaded}

\begin{verbatim}
## Warning in stac_image_collection(hum_items$features, asset_names = c("B02", :
## STAC asset with name 'SCL' does not include eo:bands metadata and will be
## considered as a single band source
\end{verbatim}

\begin{Shaded}
\begin{Highlighting}[]
\NormalTok{hum\_cube }\OtherTok{\textless{}{-}} \FunctionTok{cube\_view}\NormalTok{(}\AttributeTok{srs =} \StringTok{"EPSG:4326"}\NormalTok{,  }
                  \AttributeTok{extent =} \FunctionTok{list}\NormalTok{(}\AttributeTok{t0 =}\NormalTok{ hum\_start, }\AttributeTok{t1 =}\NormalTok{ hum\_end,}
                                \AttributeTok{left =}\NormalTok{ hum\_box[}\DecValTok{1}\NormalTok{], }\AttributeTok{right =}\NormalTok{ hum\_box[}\DecValTok{3}\NormalTok{],}
                                \AttributeTok{top =}\NormalTok{ hum\_box[}\DecValTok{4}\NormalTok{], }\AttributeTok{bottom =}\NormalTok{ hum\_box[}\DecValTok{2}\NormalTok{]),}
                  \AttributeTok{nx =} \DecValTok{1000}\NormalTok{, }\AttributeTok{ny =} \DecValTok{1000}\NormalTok{, }\AttributeTok{dt =} \StringTok{"P1M"}\NormalTok{,}
                  \AttributeTok{aggregation =} \StringTok{"median"}\NormalTok{, }\AttributeTok{resampling =} \StringTok{"average"}\NormalTok{)}

\NormalTok{S2.mask }\OtherTok{\textless{}{-}} \FunctionTok{image\_mask}\NormalTok{(}\StringTok{"SCL"}\NormalTok{, }\AttributeTok{values=}\FunctionTok{c}\NormalTok{(}\DecValTok{3}\NormalTok{,}\DecValTok{8}\NormalTok{,}\DecValTok{9}\NormalTok{)) }\CommentTok{\# mask clouds and cloud shadows}

\NormalTok{hum\_ndvi }\OtherTok{\textless{}{-}} 
  \FunctionTok{raster\_cube}\NormalTok{(hum\_col, hum\_cube, }\AttributeTok{mask =}\NormalTok{ S2.mask) }\SpecialCharTok{|\textgreater{}}
  \FunctionTok{select\_bands}\NormalTok{(}\FunctionTok{c}\NormalTok{(}\StringTok{"B08"}\NormalTok{, }\StringTok{"B04"}\NormalTok{)) }\SpecialCharTok{|\textgreater{}}
  \FunctionTok{apply\_pixel}\NormalTok{(}\StringTok{"(B08{-}B04)/(B08+B04)"}\NormalTok{, }\StringTok{"NDVI"}\NormalTok{) }\SpecialCharTok{|\textgreater{}} \FunctionTok{aggregate\_time}\NormalTok{(}\StringTok{"P3M"}\NormalTok{)}

\NormalTok{hum\_ndvi\_plot }\OtherTok{\textless{}{-}}\NormalTok{ hum\_ndvi }\SpecialCharTok{|\textgreater{}} \FunctionTok{st\_as\_stars}\NormalTok{()}

\FunctionTok{tm\_shape}\NormalTok{(hum\_ndvi\_plot) }\SpecialCharTok{+} \FunctionTok{tm\_raster}\NormalTok{(}\AttributeTok{style =} \StringTok{"quantile"}\NormalTok{) }\SpecialCharTok{+} \FunctionTok{tm\_shape}\NormalTok{(hum\_sf) }\SpecialCharTok{+} \FunctionTok{tm\_polygons}\NormalTok{()}
\end{Highlighting}
\end{Shaded}

\begin{verbatim}
## Variable(s) "NA" contains positive and negative values, so midpoint is set to 0. Set midpoint = NA to show the full spectrum of the color palette.
\end{verbatim}

\begin{verbatim}
## Legend labels were too wide. The labels have been resized to 0.49, 0.53, 0.53, 0.53, 0.53. Increase legend.width (argument of tm_layout) to make the legend wider and therefore the labels larger.
\end{verbatim}

\includegraphics{final-individual-melody_files/figure-latex/humboldt ndvi-1.pdf}

\hypertarget{humboldt}{%
\section{Humboldt}\label{humboldt}}

Humboldt County hugs the Pacific Ocean and is one county away from the
Oregon-California border, separated by Del Norte County. Humboldt is
home to cities like Eureka and Arcata which are beautiful and extremely
abundant on Redwood trees. The California Redwoods are amongst the
tallest in the world and Humboldt coutny is home to the home to them
all, making it an easy target for Bigfeet to reside.

Humboldt's NDVI is concentrated up North and more inland. Our Bigfoot
sightings did not have a lot of specific Longitudes and Latitudes, but
from the data we do have, it is prevalent that there is no distinct
correlation between Greeness in Humboldt with Bigfoot Sightings. It
seems more likely that Bigfoot Sightings are more concentrated Southwest
bound.

\begin{Shaded}
\begin{Highlighting}[]
\NormalTok{siskiyou\_sightings }\OtherTok{\textless{}{-}} \FunctionTok{filter}\NormalTok{(pnw\_sum, county }\SpecialCharTok{==} \StringTok{"Siskiyou County"}\NormalTok{)}

\NormalTok{sis\_latlong }\OtherTok{\textless{}{-}}\NormalTok{ siskiyou\_sightings }\SpecialCharTok{\%\textgreater{}\%} \FunctionTok{select}\NormalTok{(longitude, latitude) }\SpecialCharTok{\%\textgreater{}\%} \FunctionTok{na.omit}\NormalTok{()}

\NormalTok{sis\_sf2 }\OtherTok{\textless{}{-}} \FunctionTok{st\_as\_sf}\NormalTok{(sis\_latlong, }\AttributeTok{coords =} \FunctionTok{c}\NormalTok{(}\StringTok{\textquotesingle{}longitude\textquotesingle{}}\NormalTok{, }\StringTok{\textquotesingle{}latitude\textquotesingle{}}\NormalTok{))}

\NormalTok{sis\_plot2 }\OtherTok{\textless{}{-}} \FunctionTok{ggplot}\NormalTok{(sis\_sf2) }\SpecialCharTok{+} 
  \FunctionTok{geom\_sf}\NormalTok{(}\FunctionTok{aes}\NormalTok{()) }\SpecialCharTok{+} \FunctionTok{labs}\NormalTok{(}\AttributeTok{title =} \StringTok{"Siskiyou Sightings"}\NormalTok{, }\AttributeTok{x =} \StringTok{"Longitude"}\NormalTok{, }\AttributeTok{y =} \StringTok{"Latitude"}\NormalTok{)}

\NormalTok{sis\_plot2}
\end{Highlighting}
\end{Shaded}

\includegraphics{final-individual-melody_files/figure-latex/siskiyou sightings-1.pdf}

\begin{Shaded}
\begin{Highlighting}[]
\NormalTok{plot\_sis }\OtherTok{\textless{}{-}} \FunctionTok{ggplot}\NormalTok{() }\SpecialCharTok{+} 
  \FunctionTok{geom\_sf}\NormalTok{(}\AttributeTok{data =}\NormalTok{ sis\_sf, }\AttributeTok{size =} \FloatTok{1.5}\NormalTok{, }\AttributeTok{color =} \StringTok{"black"}\NormalTok{, }\AttributeTok{fill =} \StringTok{"lightblue"}\NormalTok{) }\SpecialCharTok{+} 
  \FunctionTok{ggtitle}\NormalTok{(}\StringTok{"Siskiyou Outline + Census Data"}\NormalTok{) }\SpecialCharTok{+} 
  \FunctionTok{coord\_sf}\NormalTok{() }\SpecialCharTok{+} \FunctionTok{labs}\NormalTok{(}\AttributeTok{x =} \StringTok{"Longitude"}\NormalTok{, }\AttributeTok{y =} \StringTok{"Latitude"}\NormalTok{)}
\NormalTok{plot\_sis}
\end{Highlighting}
\end{Shaded}

\includegraphics{final-individual-melody_files/figure-latex/siskiyou-1.pdf}

\begin{Shaded}
\begin{Highlighting}[]
\NormalTok{sis\_start }\OtherTok{\textless{}{-}} \StringTok{"2022{-}06{-}01"}
  
\NormalTok{sis\_end }\OtherTok{\textless{}{-}} \StringTok{"2022{-}08{-}31"}

\NormalTok{sis\_box }\OtherTok{\textless{}{-}} \FunctionTok{st\_bbox}\NormalTok{(}\FunctionTok{c}\NormalTok{(}\AttributeTok{xmin =} \DecValTok{121}\NormalTok{, }\AttributeTok{xmax =} \DecValTok{124}\NormalTok{, }\AttributeTok{ymax =} \FloatTok{42.2}\NormalTok{, }\AttributeTok{ymin =} \FloatTok{40.8}\NormalTok{))}

\NormalTok{sis\_items }\OtherTok{\textless{}{-}} 
  \FunctionTok{stac}\NormalTok{(}\StringTok{"https://earth{-}search.aws.element84.com/v0/"}\NormalTok{) }\SpecialCharTok{|\textgreater{}}
  \FunctionTok{stac\_search}\NormalTok{(}\AttributeTok{collections =} \StringTok{"sentinel{-}s2{-}l2a{-}cogs"}\NormalTok{, }\AttributeTok{bbox =} \FunctionTok{c}\NormalTok{(sis\_box), }
              \AttributeTok{datetime =} \FunctionTok{paste}\NormalTok{(sis\_start, sis\_end, }\AttributeTok{sep=}\StringTok{"/"}\NormalTok{),}
              \AttributeTok{limit =} \DecValTok{100}\NormalTok{) }\SpecialCharTok{|\textgreater{}} \FunctionTok{post\_request}\NormalTok{()}

\NormalTok{sis\_col }\OtherTok{\textless{}{-}} \FunctionTok{stac\_image\_collection}\NormalTok{(sis\_items}\SpecialCharTok{$}\NormalTok{features, }\AttributeTok{asset\_names =} \FunctionTok{c}\NormalTok{(}\StringTok{"B02"}\NormalTok{, }\StringTok{"B03"}\NormalTok{, }\StringTok{"B04"}\NormalTok{,}\StringTok{"B08"}\NormalTok{, }\StringTok{"SCL"}\NormalTok{), }\AttributeTok{property\_filter =}\NormalTok{ \textbackslash{}(x) \{x[[}\StringTok{"eo:cloud\_cover"}\NormalTok{]] }\SpecialCharTok{\textless{}} \DecValTok{20}\NormalTok{\})}
\end{Highlighting}
\end{Shaded}

\begin{verbatim}
## Warning in stac_image_collection(sis_items$features, asset_names = c("B02", :
## STAC asset with name 'SCL' does not include eo:bands metadata and will be
## considered as a single band source
\end{verbatim}

\begin{Shaded}
\begin{Highlighting}[]
\NormalTok{sis\_cube }\OtherTok{\textless{}{-}} \FunctionTok{cube\_view}\NormalTok{(}\AttributeTok{srs =} \StringTok{"EPSG:4326"}\NormalTok{,  }
                  \AttributeTok{extent =} \FunctionTok{list}\NormalTok{(}\AttributeTok{t0 =}\NormalTok{ sis\_start, }\AttributeTok{t1 =}\NormalTok{ sis\_end,}
                                \AttributeTok{left =}\NormalTok{ sis\_box[}\DecValTok{1}\NormalTok{], }\AttributeTok{right =}\NormalTok{ sis\_box[}\DecValTok{3}\NormalTok{],}
                                \AttributeTok{top =}\NormalTok{ sis\_box[}\DecValTok{4}\NormalTok{], }\AttributeTok{bottom =}\NormalTok{ sis\_box[}\DecValTok{2}\NormalTok{]),}
                  \AttributeTok{nx =} \DecValTok{1000}\NormalTok{, }\AttributeTok{ny =} \DecValTok{1000}\NormalTok{, }\AttributeTok{dt =} \StringTok{"P1M"}\NormalTok{,}
                  \AttributeTok{aggregation =} \StringTok{"median"}\NormalTok{, }\AttributeTok{resampling =} \StringTok{"average"}\NormalTok{)}

\NormalTok{S2.mask }\OtherTok{\textless{}{-}} \FunctionTok{image\_mask}\NormalTok{(}\StringTok{"SCL"}\NormalTok{, }\AttributeTok{values=}\FunctionTok{c}\NormalTok{(}\DecValTok{3}\NormalTok{,}\DecValTok{8}\NormalTok{,}\DecValTok{9}\NormalTok{)) }\CommentTok{\# mask clouds and cloud shadows}

\NormalTok{sis\_ndvi }\OtherTok{\textless{}{-}} 
  \FunctionTok{raster\_cube}\NormalTok{(sis\_col, sis\_cube, }\AttributeTok{mask =}\NormalTok{ S2.mask) }\SpecialCharTok{|\textgreater{}}
  \FunctionTok{select\_bands}\NormalTok{(}\FunctionTok{c}\NormalTok{(}\StringTok{"B08"}\NormalTok{, }\StringTok{"B04"}\NormalTok{)) }\SpecialCharTok{|\textgreater{}}
  \FunctionTok{apply\_pixel}\NormalTok{(}\StringTok{"(B08{-}B04)/(B08+B04)"}\NormalTok{, }\StringTok{"NDVI"}\NormalTok{) }\SpecialCharTok{|\textgreater{}} \FunctionTok{aggregate\_time}\NormalTok{(}\StringTok{"P3M"}\NormalTok{)}

\NormalTok{sis\_ndvi\_plot }\OtherTok{\textless{}{-}}\NormalTok{ sis\_ndvi }\SpecialCharTok{|\textgreater{}} \FunctionTok{st\_as\_stars}\NormalTok{()}

\FunctionTok{tm\_shape}\NormalTok{(sis\_ndvi\_plot) }\SpecialCharTok{+} \FunctionTok{tm\_raster}\NormalTok{(}\AttributeTok{style =} \StringTok{"quantile"}\NormalTok{) }\SpecialCharTok{+} \FunctionTok{tm\_shape}\NormalTok{(sis\_sf) }\SpecialCharTok{+} \FunctionTok{tm\_polygons}\NormalTok{()}
\end{Highlighting}
\end{Shaded}

\begin{verbatim}
## Variable(s) "NA" contains positive and negative values, so midpoint is set to 0. Set midpoint = NA to show the full spectrum of the color palette.
\end{verbatim}

\includegraphics{final-individual-melody_files/figure-latex/Siskiyou ndvi-1.pdf}
\# Siskiyou

Siskiyou County is home to the most biodiverse organisms in California.
They are very important and different as the Siskiyou land is home to
past volcanic ash and erosion. Therefore the soil here is very different
from other Californian soil and Mediterranean climate as it touches the
border of Oregon.

As we can see, the Bigfeet sightings in Siskiyou are concentrated in the
upper and lower left hand side of the coutny as well as one in the deep
forests of the lower right region. These areas are more consistently
green than the left side, however nothing truly stands out as a
connection between those areas and Bigfeet sightings.

\hypertarget{conclusion}{%
\subsection{Conclusion}\label{conclusion}}

Bigfeet are rare and seen to be scary beasts that are seen lurking at
night, but as Laura Krantz has figured out: it doesn't matter if Bigfoot
is real or not because either way, the Earth has been in need of some
love from its inhabitants. We have drifted too far into the world of
technology, that the environment no longer suits our needs, as long as
we can advance in technology.

If Bigfoot is discovered, there would be a triumphant roar for many of
us. Especially conservationists as they try and figure out more ways
that humans can be proactive in the conservation of the Earth. A rare
organism and identity like Bigfoot would likely cause the areas of the
Pacific Northwest to reign in government protection.

Bigfoot are mammals like humans, they may outlive us without ever being
discovered, but their organism peers deserve a suitable home and
foundation away from chaos just like humans.

\end{document}
